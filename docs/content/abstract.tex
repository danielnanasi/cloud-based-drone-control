%\pagenumbering{roman}
%\setcounter{page}{1}

\selecthungarian

%----------------------------------------------------------------------------
% Abstract in Hungarian
%----------------------------------------------------------------------------
\chapter*{Kivonat}\addcontentsline{toc}{chapter}{Kivonat}

A diplomaterv a mai felhőrendszerekről és azokban drón irányítási megoldásaiban mélyed el.
Megmutatja a mai rendszerek felhőrendszerek szolgáltatás alapú technológiáit és bevezeti az olvasót a konténeralapú szolgáltatásmenedzsmentbe. Leírást ad a felhasználói igényekről és a felhőrendszereknek a jövőbeli fejlesztési lehetőségeiről. Áttekintést ad a közösségi fejlesztések által nyújtott nagy méretű erőforrás menedzsment megoldásoknak a mikéntjére, azok hibájára és bemutatja tesztelésüket. A dolgozat ismerteti és feldolgozza a mai robotirányítási megoldásokat, különös tekintettel a drónirányításra, azoknak ipari felhasználásáról és megvalósíthatóságáról. Betekintést ad miként valósítható meg egy 5G kommunikációra alapuló drónirányításra tervezett felhőrendszer. Megmutat egy nagyobb számú drónt központi konténer alapú vezérlő és jelfeldolgozó rendszer megvalósítást. Továbbá kifejti ezeknek a tervezési és implementációs lépéseit és a távvezérlési funkciók dinamikus elhelyezését megvalósító módszert. Egy emulált környezetben megvizsgálja több drón távoli vezérlésének a sajátosságait. Ennek tükrében összeveti a drónvezérlés QoS (Quality of Service) igényeit és a megvalósított rendszer által biztosított feltételrendszert, értékelve a megvalósítás részleteit. Bemutat egy  szoftver megvalósítást, amely a felhőrendszerben biztosítja az elvárt QoS-t, végül mérés alapú vizsgálatok segítségével értékeli azt a kialakított tesztrendszerben. Összefoglalja az elvégzett munkát, a szimuláció eredményeit és a QoS feltételrendszert a kialakított tesztrendszerben.


\vfill
\selectenglish


%----------------------------------------------------------------------------
% Abstract in English
%----------------------------------------------------------------------------
\chapter*{Abstract}\addcontentsline{toc}{chapter}{Abstract}

The Thesis focuses on today’s cloud systems and the control solutions of drones. It overviews the current cloud technologies, with specific focus on container-based service management, and describes the specific service requirements. The Thesis presents the current trends in remote device control, detailing the drone control aspects and its feasability.  It provides insight into how to implement a cloud system designed for remote control based on a 5G communication system. It presents the design aspects of a system of larger number of robots, detailing the operation based on container-based remote control system. Additionaly, it presents the design and implementation steps, as well as an algortihm for the dynamic placement problem of the remote control functions. It investigates the properties of remote control of drones in an emulated environment. It analyses the Quality of Service (QoS) requirements of such operation, and discusses the implementation details. The thesis describes in detail the software implementation of the above proposals, and the cloud deployment details. Finally, it evaluates it by measurements done in the test system.

\vfill
\selectthesislanguage

\newcounter{romanPage}
\setcounter{romanPage}{\value{page}}
\stepcounter{romanPage}