%\pagenumbering{roman}
%\setcounter{page}{1}

\selecthungarian

%----------------------------------------------------------------------------
% Abstract in Hungarian
%----------------------------------------------------------------------------
\chapter*{Kivonat}\addcontentsline{toc}{chapter}{Kivonat}

A diplomaterv a mai felhőrendszerekről és azokban robot irányítási megoldásaiban mélyed el.
Megmutatja a mai rendszerek felhőrendszerek szolgáltatás alapú technológiáit és bevezeti az olvasót a konténeralapú szolgáltatásmenedzsmentbe. Leírást ad a felhasználói igényekről és a felhőrendszereknek a jövőbeli fejlesztési lehetőségeiről. Áttekintést ad a közösségi fejlesztések által nyújtott nagy méretű erőforrás menedzsment megoldásoknak a mikéntjére, azok hibájára és bemutatja tesztelésüket. A dolgozat ismerteti és feldolgozza a mai robotirányítási megoldásokat, kifejezett prioritást tekintve a drón irányításnak, azoknak ipari felhasználásáról és megvalósíthatóságáról. Betekintést ad miként valósítható meg egy 5G kommunikációra alapuló robotirányításra tervezett felhőrendszer. Megmutat egy nagyobb számú robotot központi konténer alapú vezérlő és jelfeldolgozó rendszer megvalósítást. Továbbá kifejti ezeknek a tervezési és implementációs lépéseit és a távvezérlési funkciók dinamikus elhelyezését megvalósító módszert. Szimulációt mutat nagyszámú robotvezérlés és annak kamerajelének feldolgozására a kialakított rendszerben. Ennek tükrében összeveti a drónvezérlés QoS (Quality of Service) igényeinek feltételeit és a megvalósított rendszer funkcióinak ezen feltételrendszerre szabott tervezési megoldást ad a felhő rendszeren belül!
Összefoglalja az elvégzett munkát, a szimuláció eredményeit és a QoS feltételrendszert a kialakított tesztrendszerben.


\vfill
\selectenglish


%----------------------------------------------------------------------------
% Abstract in English
%----------------------------------------------------------------------------
\chapter*{Abstract}\addcontentsline{toc}{chapter}{Abstract}

The thesis delves into today’s cloud systems and the control solutions of individual robots.
It demonstrates today’s systems with cloud-based service technologies and introduces readers to container-based service management. Provides a description of user needs and cloud systems. Overview and a community development of large-scale resource management solutions is a way of failing and presenting their testing. The thesis describes and processes robot control solutions with explicit priority for drone guidelines, industrial use, and applicability. Provides insight into how to implement a cloud system designed for robotic control based on a 5G communication system. It focuses for larger number of robots in the operation of a central container-based control and signal processing system. In addition, it shows the design and implementation steps as well as the dynamic placement of remote control functions in the system design methodology. It analyses results of simulations in a system created to process a large number of robot controls and their camera signals. In order to improve quality, in order to facilitate the use of QoS (Quality of Service) and to facilitate the operation of the established system, the conditionalities usually becomes available.
Summarize the work, simulation, development and QoS conditions of the test system.

\vfill
\selectthesislanguage

\newcounter{romanPage}
\setcounter{romanPage}{\value{page}}
\stepcounter{romanPage}