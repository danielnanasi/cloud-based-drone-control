\chapter{Bevezetés}

Ebben a fejezetben betekintést nyújtok az olvasó számára a feladatról, mint problémáról, hogy az miért is indokolt és milyen iparágakat érinthet, mi a pontos célja a feladatnak és mik a főbb kritériumok. Tisztázni szeretném a képet az olvasóban, hogy mik is a projekt előzményei amely alapján kialakult a feladatspecifikáció

\section{Előzmények}
dfsfds
\section{A feladat célja}
fdsfsd
\section{Feladat indokoltsága}
fdsfsdf


\section{Használt kifejezések}
A hálózati-, virtualizációs- és robotiparban rengeteg rövidítés, mozaikszó és kifejezés létezik, amit főként csak azok ismernek, akik
jelentősebben mélyültek el ezen iparág területén. Két csoportra bontom a diplomatervemhez használt kifejezéseket:
\begin{enumerate}
	\item Azon szakmai kifejezések, amik elterjedtek a mérnöki szakmákban és mondjuk a BME VIK tetszőleges hallgatója, bármely, nem infokommunikációs szakosodás mellett is nagy valószínűséggel ismer és nem kell ismertetnem a dolgozatomban. Pár példa a kategóriában, amikre külön nem térek ki, nem oldom fel a szövegben, ilyenek az IPv4, NAT, hálózati réteg, HTTP, CPU, for ciklus.
	\item 
\end{enumerate}
 Alapvetően a témához kapcsolódó kifejezéseket,
amik nem elterjedtek a szakmában, azokat a dolgozat folyamán definiálom. Az alapvető
hálózati rövidítéseket/kifejezéseket pedig definiálás nélkül használom. Azokat a hálózati
kifejezéseket tekintettem alapvetőnek, amiket egy nem infokommunikáció specializációra
szakosodott mérnök évfolyamtársamnak is ismernie kell (például IPv4, NAT, hálózati réteg,
HTTP, CPU, for ciklus). Azokat az alapvető kifejezéseket, amik a virtualizált hálózatok
témájában elengedhetetlenek, itt definiálom: