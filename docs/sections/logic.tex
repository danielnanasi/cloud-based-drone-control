\chapter{Drónkapcsoló állomás Kubernetes felhőben}
Ebben a fejezetben, a kialakított K3S alapú drón kiszolgáló központot fejlesztem tovább, úgy hogy valamilyen QoS feltételeket valósítson meg. Megnézzük, hogyan választottam technológiát a szoftver architektúráját, részleteit. Tárgyaljuk, hogy milyen architekturális változást kell végeznünk ahhoz, hogy javítsuk a válaszidőt. Megmutatom a Node váltási lehetőségeket és becsléseket teszek a hatékonyságukra. \\

\noindent
A fejlesztendő szoftver a drón közelébe tervezett, tehát nem a felhőben fog futni, hanem a drónok központi irányításának a kiegészítése. E szerint futtatását azon a gépen célszerű végezni, mely a drónnal is egy hálózaton/környezetben van.

\section{Felhasznált technológia}
Olyan technológiában kell megkezdeni a fejlesztést, amely lehetőséget biztosít egyszerű kezelésére az eddig felállított rendszernek és környezetnek. Szükséges belevinni modularitást és sklázhatóságot, továbbfejlesztési lehetőség érdekében. Tehát a felhasználó számíthat rá, hogy nagyobb számú drón vagy robotvezérlést is támogat a rendszer. \\

\noindent
Mivel egy QoS megvalósításáról beszélünk, amelynek paraméterei változhatnak a szükségletek függvényében, ezért az implementáció részeként csak a logikai működést szükséges megvalósítani, az egyes kritériumok változtathatóak. A bevezetőben szó esett különböző jövőbeli automatikus drónfelhasználásnak, melyben volt példa élő szórakoztatóipari fellépésre és mezőgazdasági munkára is. Ezen iparágakban teljesen más QoS feltételeket szabnak meg. Míg egy élő fellépésben nagyon hamar belerondíthat a show-ba pár késleltetésből félreszinkronizált drón, addig a mezőgazdaságban inkább hosszú idejű stabilitásnak kellhet megfelelni. Ezért az algoritmus megalkotásánál paraméterekben fogunk dolgozni, melyeket a felhasználás definiál.

\subsection{Python}
A szakmai előítéleteim alapján már sejtettem, hogy a Python lesz az a nyelv, amelynek a legtöbb támogatása van a natív felhőtechnológiákhoz. A Python egy értelmezett, magas szintű és általános célú programozási nyelv. A Python tervezési filozófiája hangsúlyozza a kód olvashatóságát a jelentős szóköz jelentős felhasználásával. Nyelvi konstrukcióinak és objektum-orientált megközelítésének célja, hogy segítsen a programozóknak világos, logikus kódot írni kis és nagy projektekhez. A Python dinamikusan be van írva és automatikus garbage-collector-ral működik. Támogatja a több programozási paradigmát, beleértve a strukturált (különösen az eljárási), az objektum-orientált és a funkcionális programozást. A Python-t átfogó szabványos könyvtárának köszönhetően gyakran "csomagokkal együtt" használt nyelvként írják le. \cite{pwiki} \\

\noindent
Amely lehetőségek miatt kifejezetten a Python technológiát választottam azok az alábbiak:
\begin{itemize}
	\item Kubernetes API támogatás
	\item Docker API támogatás
	\item Könnyű naplózás, időalapú feljegyzés
	\item Gyenge objektumorientált támogatás, a modularitás és a tiszta kód miatt
	\item Egyszerűség, gyengén típusosság
\end{itemize}

\noindent
A dolgozathoz mellékelt könyvtár \emph{switchingCenter} nevű könyvtárban található a Python projekt implementáció. A könyvtárban található még egy \emph{requiments} felsorolás, mely a használt Python könyvtárakat sorolja fel azzal a verzióval, mellyel teszteltem is. Ezen kiegészítések telepítése a Pip programmal lehetséges az adott környezetben (\ref{lst:pip}. lista).

\begin{lstlisting}[caption={Python könyvtárak telepítése}, label={lst:pip}]
pip install - r requiments.txt
\end{lstlisting}

\noindent
A programhoz tartozó beállításokat egy globális fájlba helyeztem, melyet a Bash és a Python program egyaránt fel tud használni, utóbbi a \emph{ConfigObj} konfiguráció illesztő modullal.

\subsection{Kubernetes könyvtár alkalmazása}
Nyílt forráskódú, stabil verziójú Kubernetes Python kliens létezik, amelyet felhasználtam a fejlesztés során. \cite{kubpy} Mint a kliens főoldalán is látható, Kubernetes API-t széleskörűen támogatja, így az általam felhasznált Kubernetes API technológiákat is. \\

\noindent
A jelenlegi környezetben a program felhasználja a Multipass program parancsait a Bash-en keresztül, ami nem túl kifinomult mérnöki kivitelezés, azonban a Multipass-nak még nincs Python kliense, illetve későbbi rendszerek esetén könnyen átfejleszthető másra. A külső program meghívására a \emph{subprocess} modult használtam, mely segítségével kértem le a master VM-ről a K3S tokent és a master IP-jét. Utána a kapott információkból fel tudtam építeni a Kubernetes authentikációt, megadva a Multipass által generált privát kulcsot is (\ref{lst:kubclient}. lista).

\begin{lstlisting}[caption={Kubernetes kliens felépítése}, label={lst:kubclient}]
aToken= subprocess.check_output(["multipass", "exec", self.config_parser.get('master'), "--", "kubectl describe secret $SECRET_NAME | grep -E '^token' | cut -f2 -d':' | tr -d " ")"]).decode('utf-8')
master_ip= subprocess.check_output(["multipass", "list", "|", "grep", self.config_parser.get('master'),'grep -oE \b([0-9]{1,3}\.){3}[0-9]{1,3}\b']).decode('utf-8')

aConfiguration = kubernetes.client.Configuration()
aConfiguration.verify_ssl = False
aConfiguration.host = "https://"+master_ip+":443"
aConfiguration.api_key = {"authorization": "Bearer " + aToken}
aApiClient = kubernetes.client.ApiClient(aConfiguration)

self.kApi = client.CoreV1Api(aApiClient)
\end{lstlisting}

\subsection{Docker könyvtár alkalmazása}
A \emph{Docker SDK for Python} segítségemre volt, hogy a Drónokat elérjem és parancsokat adhassak ki a konténeren belül. Az SDK nagyon egyszerű lehetőséget nyújtott kapcsolódni a környezethez, kifejezetten ha a szoftver is ugyanabban a fut. \\

\noindent
Definiáltam egy Drone osztályt, amely példányosításával egy drónhoz kapcsolódik a program, melyet tárol és fenntartja a kapcsolatot. Az inicializálás a drón száma alapján történik, hasonlóan a korábbiakhoz, az azonosítóból kiszámoljuk a portokat, mely kezdőportokat a konfigfájlból olvasunk ki (\ref{lst:droneconnect}. lista). A konténert a konvencionális nevezéktanon keresztül ismerjük fel (drone-1, drone-2, ...), mely konténer lekérése után bármilyen Docker parancsot végre tudunk hajtani a \emph{self.drone} elnevezésű kliens segítségével.

\begin{lstlisting}[caption={Drón konténeréhez csatlakozás}, label={lst:droneconnect}]
class Drone():
	def __init__(self, drone_id, node_ip):
		self.config_parser = ConfigObj('../config/config')
		self.drone_id=int(drone_id)
		self.node_ip=node_ip
		self.docker_client = docker.from_env()
		self.mavlink_port = int(self.config_parser.get('MAVLINK_START_PORT'))+drone_id-1

		self.drone= self.docker_client.containers.get("drone-"+str(self.drone_id))
\end{lstlisting}

\subsection{Naplózás}
Szoftveremben nagyon fontos a naplózás, melyet a \emph{logger} modullal valósítottam meg. A QoS miatt minden egyes tevékenységet, amit a program elvégez nyers időbélyeg formátumban írom ki, hogy méréseket lehessen végezni a tevékenységek között (\ref{lst:logconf}. lista).

\begin{lstlisting}[caption={Naplózás beállítása}, label={lst:logconf}]
logging.basicConfig(filename='switchCenter.log', level=logging.INFO, format='%(asctime)s %(levelname)-8s %(message)s')
\end{lstlisting}

\section{Kivitelezési lehetőségek és várható kapcsolási idők}

Tehát van egy automatizáltan felépített rendszerünk Kubernetes-szel és változtatható mennyiségű szimulált drónnal. És már a szoftver is készen áll, hogy bármely automatizált szabályokat figyelhessünk és bármilyen módosítást végrehajtsunk valós időben a rendszeren. De mik is azok a feltételek, amelyek egy QoS biztosítanak egy ilyen drónokat irányító felhőrendszerben? Definiálom ezeket egy drónra:
\begin{itemize}
	\item A drón és a Kubernetes közötti válaszidő nem haladhat meg egy előre definiált maximális időt,
	\item A drón és a Kubernetes közötti sávszélesség nem lehet kisebb egy előre definiált minimális sávszélességnél, ez a sávszélesség szétbontható egy irányításra meghatározott biztonsági és egy videófolyam sávszélességre,
	\item Ha ezen feltételeknek nem felel meg az adott konstrukció, végezzen változtatást,
	\item Ha nem létezik olyan változtatást elvégezni, amellyel megfelelnek a feltételek, akkor érjen véget a program és jelezze, hogy véget ért a QoS feltételek melleti üzemelés, mely esetnen létrejöhet valamilyen biztonsági procedúra, például biztonságos leszállás.
\end{itemize}

\noindent
A következőkben megnézzük, hogy mi az a változtatás amivel javítani lehet a kapcsolaton. Jelen rendszerben egy 3 Node-os felhőrendszerről beszélünk, amely lehet sokkal szélesebb is, a K8S határain belül. Az implementációnál figyelembe vettem a változtatható Node mennyiséget. Ha valamilyen kritériumnak nem felel meg a kapcsolat, akkor egyféleképp változtathatjuk a kapcsolatot, áthelyezzük a háttérszolgáltatást másik Node-ra. Ebben a lokális virtuális rendszerben ez nem tűnik megoldásnak, de egy nagyobb ipari klaszter ezetén a távolságokból adódóan Node váltással érhetünk el jobb eredményt, biztosíthatjuk rendszerünket túlterheltség és véletlen hiba létrejötte esetén. Tehát akkor azt kell megnéznünk, mely módokon lehet kivitelezni.

\subsection{Node váltás és új címhirdetés}
Első ilyen megközelítés, hogy ha már automatizáltan fel tudjuk építeni a Roscore-t, akkor legyen a változtatás az, hogy új Node-ra deploy-oljuk a Roscore-t és közöljük a drónnal az új elérési címet. Nézzük meg mi ennek a pontos menete:
\begin{itemize}
	\item Észrevesszük a hibát és reagálunk ($t_{r}$)
	\item Kiválasztunk egy megfelelő Node-ot ($t_{s}$)
	\item Deploy-oljuk a Roscore-t az új Node-on ($t_{d}$)
	\item Új címet adunk meg a drónnak ($t_{a}$)
\end{itemize}
Összesen $t = t_r + t_s + t_d + t_a$ idő. Azt fontos megjegyezni, hogy a $t_r$ reagálási idő konfigurálható vagyis az, hogy milyen időközönként nézzük, amiből tudjuk, hogy a reagálási idő várható értéke az ellenőrzési periódusidő fele lesz.

\subsection{Node váltás proxy mögött}
Nézzünk meg egy másik megközelítést, amikor meg szeretnénk spórolni $t_a$ címadási időt és kihelyezünk egy proxy-t amely megváltoztatja a mögöttes Node-ot miután a Roscore-t deploy-oltuk az új Node-ra. A probléma továbbra is fennáll, hogy ha egy Node-on fut a Roscore, akkor azt kívülről az elérhető portok nagy részén el kell tudnunk érni. A Kubernetes Service API működhetne ilyen proxy-ként, azonban nem fogja biztosítani a többi portot LoadBalancer nélkül. Sima Kubernetes proxy-val azonban elméletben meg lehet oldani (lásd dokumentáció \cite{proxy}). Azonban ezzel a megoldással létrejönne a proxy kreálás/változtatás ideje, illetve az új proxy-n keresztül hitelesítést is kell végeznünk minden alkalommal. 

\subsection{Széleskörű rendelkezésre állás}
A legtöbb időt jelentő $t_d$ deploy időt, ami méréseim szerint 2 és 5 másodperc között van, akkor iktathatjuk ki, hogyha nem csak az aktív Node-on áll rendelkezésre az adott drónnak a Roscore, hanem legalább egy backup Node-on. Ezzel a verzióval ugyan átlagosan több erőforrást foglalunk, azonban optimalizálhatjuk a váltási időt, ami javít a QoS-en.

\section{Megvalósítás: DaemonSet}
Kiderült, hogy a legoptimálisabb QoS megoldása az lehet, ha minden worker Node-on rendelkezésre áll a Roscore minden drón számára. Ezt a DaemonSet Kubernetes API-val lehet megvalósítani. \\

\noindent
A DaemonSet biztosítja, hogy az összes (vagy néhány) Node futtassa a Pod másolatát. Amint  Node-ok hozzáadódnak a klaszterhez, a Podok automatikusan létrejönnek az új Node-okon. Amint a Node-okat eltávolítják, ezeket a Podokat a Kubernetes Garbage-Collectora gyűjti össze. A DaemonSet törlése megtisztítja az általa létrehozott Podokat. Egyszerű esetben minden démontípushoz egy, minden csomópontot lefedő DaemonSet-et használnak. A bonyolultabb beállítás több DaemonSet-et is használhat egyetlen démontípushoz, de különböző jelzőkkel és / vagy különböző memória- és CPU-kérelmekkel a különböző hardvertípusokhoz. \cite{daemonset} \\

\noindent
A Kubernetes Service API helyett használom a megvalósítandó ötletben a DaemonSet-et a következőképpen (\ref{lst:daemonset}. lista). A konténer indítás és a változók ugyanazok maradtak, mint a Service-es megvalósításban, az \emph{affinity} résszel egészítettem ki a yaml-t. Itt \emph{hostname} alapján szűröm ki a Node-okat, azokon a Node-okon fog elindulni a Roscore, melyek nem tartalmazzák a master nevet.

\begin{lstlisting}[caption={DaemonSet megvalósítás, Roscore minden worker-en}, label={lst:daemonset}]
apiVersion: apps/v1
kind: DaemonSet
metadata:
	name: roscore-$(DRONE_IDENTIFIER)
spec:
	selector:
		matchLabels:
			app: roscore-$(DRONE_IDENTIFIER)
	template:
		metadata:
			labels:
				app: roscore-$(DRONE_IDENTIFIER)
		spec:
			affinity:
				nodeAffinity:
					requiredDuringSchedulingIgnoredDuringExecution:
						nodeSelectorTerms:
						-	matchFields:
							-	key: metadata.name
								operator: NotIn
								values:
								-	master
			hostname: roscore-$(DRONE_IDENTIFIER)
			containers:
			-	name: roscore-$(DRONE_IDENTIFIER)
				image: alpineros/alpine-ros:noetic-ros-core 
				env:
				-	name: ROS_IP
				valueFrom:
					fieldRef:
						fieldPath: status.podIP
			-	name: ROS_MASTER_URI
				value: "http://$(ROS_IP):$(MAVLINK_PORT)"
			args: ["roscore", "-p", "$(MAVLINK_PORT)"]
		hostNetwork: true
\end{lstlisting}

\section{Mérési adatok kinyerése}
A rendszer automatikus működéséhez szükséges megtudnunk az aktuális sávszélességet és válaszidőt a drón és a node között. Mivel a szoftver a drón közelében fut, ezért a jelenlegi szimulált rendszerhez implementálom, azonban ezek a függvények változtathatóak lesznek. A program a konténerből tud kiadni alap Linuxos parancsokat, így a sávszélességet iperf programmal (\ref{lst:bw}. lista), a válaszidőt ping programmal (\ref{lst:ping}. lista)  nyertem ki. Ha nem sikerülne elvégezni a program futását, akkor visszatér egy magas értékkel, melyet lemér mégegyszer a váltás előtt. Ping esetén eldönthetjük, a függvény paraméterében, hogy hány ICMP packet átlagát szeretnénk venni.

\begin{lstlisting}[caption={Sávszélesség megállapítása a konténer és a node között}, label={lst:bw}]
def getBW(self, node_ip):
	iperf="iperf -s "+node_ip
	(exit_code,output)= self.drone.exec_run(iperf)
	outputStr = output.decode('utf-8')
	for line in outputStr.split('\n'):
		if 'window size' in line:
			bw=line.split(' ')[4]
			return float(bw)
	else:
		return 1000
\end{lstlisting}

\begin{lstlisting}[caption={Késleltetés megállapítása a konténer és a node között}, label={lst:ping}]
def getConnectionDelay(self, node_ip, n=1):
	pingCommand="ping "+node_ip+" -c "+str(n)
	(exit_code,output)= self.drone.exec_run(pingCommand)
	outputStr = output.decode('utf-8')
	for line in outputStr.split('\n'):
		if 'min/avg/max/mdev' in line:
			avg=line.split('/')[5]
			return float(avg)
	else:
		return 1000
\end{lstlisting}




