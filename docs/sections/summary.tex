\addcontentsline{toc}{chapter}{Összegzés}
\chapter*{Összegzés}
A dolgozatban megnéztem mire hasznának ma tömeges robotos irányítást, specifikusabban kitekintettünk, hogy mely iparágakban használhatóak a drónok, mire és hogyan használják. Megnéztem mire jó a konténerizáció, elmerültünk a konténer alapú felhőrendszerek világában, összehasonlítottuk a Docker Swarm-ot, a Mesos-t és a Kubernetest. Átnéztem a robot- és drónirányítással kapcsolatos szoftvereket, hogy mik a lehetőségek, hogyan és minek szükséges jól együttműködni egy fizikai vagy egy szimulált drón irányításához. Megnéztem, hogyan lehet nagyobb számú drónt szimulálni. Megvalósítottam két tesztet, konténerekkel való drónirányítás és jelfeldolgozás tesztjét és különböző VM-ekről drónszimulálás tesztet. Továbbá megnéztem, hogyan tudjuk megbecsülni nagyszámú drónkiszolgálásnak a késleltetését. Elkészítettem egy automatikusan felépülő virtuális Kubernetes rendszert. Ezen rendszeren teszteltem a Kubernetes API széles palettáján több drónvezérlési architektúrát. Definiáltam egy QoS feltételrendszert, majd kiválasztottam és módosítottam azt az architektúrát amely ennek a legjobban megfelel. Megterveztem és implementáltam egy kiegészítőszoftvert, amely folyamatosan figyeli a kapcsolatot a drón és a felhőrendszer között, amennyiben tud javítani a kapcsolaton ezt megteszi, ha kritikus állapotba lép és nem tud javítani, akkor pedig figyelmezteti a felhasználót. Implementáltam egy algoritmust, amely a távoli drónvezérlés során szükséges videó folyamok és vezérlési adatok együttes minőségbiztosítását végzi. Méréseket végeztem a különböző részfeladatokról és terheltségről.