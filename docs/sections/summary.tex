\addcontentsline{toc}{chapter}{Összegzés}
\chapter*{Összegzés}
A tanulmányban megnéztük mire hasznának ma tömeges robotos irányítást, specifikusabban kitekintettünk, hogy mely iparágakban használhatóak a drónok, mire és hogyan használják. Megnéztük mire jó a konténerizáció, elmerültünk a konténer alapú felhőrendszerek világában, összehasonlítottuk a Docker Swarm-ot, a Mesos-t és a Kubernetest. Átnéztük a robot- és drónirányítással kapcsolatos szoftvereket, hogy mik a lehetőségek, hogyan és minek szükséges jól együttműködni egy fizikai vagy egy szimulált drón irányításához. Megnéztük, hogyan lehet nagymennyiségő drónt szimulálni. Megnézhettünk két tesztet, konténerekkel való drónirányítás és jelfeldolgozás tesztjét és különböző VM-ekről drónszimulálás tesztet. Továbbá megnéztük, hogyan tudjuk megbecsülni nagyszámú drónkiszolgálásnak a késleltetését. Végül pedig javasoltunk egy algoritmust, amely a távoli drónvezérlés során szükséges videó folyamok és vezérlési adatok együttes minőségbiztosítását végzi.