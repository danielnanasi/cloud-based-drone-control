\chapter{Távoli Robotvezérlés Optimalizálása}

Ebben a fejezetben a távoli robotvezérléshez használt algoritmust ismertetem. Az algoritmus célja, hogy a drónok vezérléséhez biztosítsa a QoS feltételeit, amit jelen esetben két paraméterben határozunk meg: sávszélesség és válaszidő. Ennek a két paraméternek a kontrollálásával biztosítható a videó folyam minősége a drón és a Kubernetes Node-ok között. A Kubernetes klaszter worker node-jai (rendelkezésre álló szerverek) erőforrás készletének függvényében a videófeldolgozás és a drón irányítása áthelyezhető (váltható, migrálható) az optimális teljesítmény érdekében. Az algoritmus rendszeresen méri az elérhető erőforrásokat, proaktívan vált a szolgáltatás zökkenőmentes fenntartása érdekében. \\

\noindent
Tehát van valamennyi Node-unk amik között a drónok vezérlését áthelyezhetjük, az algoritmus úgy vált időközönként, hogy a késleltetést optimalizálja és biztosítja a sávszélességet is. Ha a sávszélességet nem lehet biztosítani, akkor egy ALARM állapotba fut, ekkor a drón biztonságosan leszáll és azonnal értesíti a felhasználót. \\

\noindent
Az algoritmus bemenete tehát a Node-ok IP címének halmaza, amin keresztül kapcsolódik a drón a központhoz, a minimum sávszélesség a videó folyamra felé (vbw), a minimum sávszélesség a kontroll kommunikációra (cbw), ellenőrzések között eltelt időintervallum (T) és a jobb válaszidő esetén az az arány ami esetén vált (R). Az algoritmus a teljes üzemidő alatt fut T időközönként, a kimenete pedig a videó folyam vezérlése és a virtuális funkciók lokációjának a meghatározása, hálózati erőforrás vezérlése.

\begin{lstlisting}[caption={Az optimalizáló algoritmus}]
ALARM := false
while(!ALARM):
	migrationTrigger := false
	actLat = checkActualLat()
	(minLat, minNode) = MIN{checkAllNodeLat()}
	actBW = checkActualBW()
	moreBandwith = isThereNodeWithBetterBW()
	IF (minLat / actLat > R AND morBandWith):
		migrationTrigger = true
		nextNode = minNode
	criticalState = IF (cbw + vbw > actBW)
	IF (migrationTrigger OR criticalState):
		bestNode = findBestNode()
		bBW = checkActualBW(bestNode)
		IF (bBW < cbw + vbw):
			ALARM = true
		ELSE:
			swithTo(bestNode)
	sleep(T)
\end{lstlisting}

Az algoritmusban használt migrationTrigger boolean változó-t igazra áll, hogyha a az algoritmus váltana egy jobb node-ra, az ALARM változó pedig igazra áll, ha baj van és le kell állítani biztonságosan a drónt. A minLat változóba tárolom a késleltetés minimumát, az actLat-ba pedig az aktuális node késleltetését. Hasonlóan az actBW az aktuális sávszélesség, a bestNode pedig az a node amelyikre vált az algoritmus.